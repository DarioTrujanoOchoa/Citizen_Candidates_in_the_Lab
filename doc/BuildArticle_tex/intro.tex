\section{Introduction}

    The citizen-candidate model (Osborne and Slivinski 1996, Besley and Coate 1997) is one of the few established approaches to endogenizing the number and the identity of political candidates and proposals in elections. In this environment a society of agents with publicly known preferences in some policy space has to decide on a common policy. Crucially, only alternatives explicitly proposed (nominated) by somebody shall be presented to voters and the nomination decision is strategic: citizens choose to nominate themselves, based on their predicted impact on the policy outcome, the cost of running for office and benefits accruing to office-holders. Once the set of candidates is fixed, the entire society votes and the elected candidate implements his/her favorite policy (as the individual preferences are public, candidates cannot commit to implementing any policy at variance with their ideal).
    Unfortunately, the citizen-candidate model is not easy to test using available electoral data, as it heavily relies on exact public knowledge of the the policy preferences of potential candidates, even those who may never be nominated in the actual election. Predictions of the model are, furthermore, dependent on parameters (such as the cost of running for office and the benefits of holding it) that might be difficult to measure empirically and even harder to exogenously vary in real political systems. A direct test of the model's prediction for the differential impact of different electoral systems is further complicated by the relative rarity of electoral system changes. The substantial multiplicity of equilibria for many parameter values in the model makes designing a satisfactory empirical test even harder.
    Many of the problems with testing the citizen-candidate model in the field can be overcome in an experimental lab. Thus, an experimentalist would have no difficulty varying office-holder benefits or nomination costs, changing the distribution of citizens in the policy space or even the electoral system. The hardest challenge is presented by the model's inherent equilbrium multiplicity for most "interesting" paramter values. Still, in the lab it is also possible to design environments that minimize this problem, allowing explicit tests of the model predictions.
    

    Surprisingly, in the more than twenty years since the publication of the original theoretical papers there has been little work on trying to test the model experimentally. The experimental literature on candidate behavior in elections has concentrated on candidate location decisions.
    <footnote>See, for instance, the early work by McKelvey and Ordeshook (1982) on two-candidate competition in environments with and without Condrocet winners, or a study by Aragonès and Palfrey (2004) on policy platform choice by candidates of different quality.
      
    However there has been comparatively little research on candidate entry. In fact, Palfrey (2016) in his survey of the field, noted that as of that moment he was aware of only two experimental studies concentrating on entry by policy-motivated candidates in this framework. Though an important advance, for being the first to attempt a laboratory testing of the model, Cadigan (2005) is somewhat limited in scope. It reports results of 2 treatments of an adaptation of the citizen-candidate model that are distinguished by the value of the cost of nomination parameter. In the high-cost treatment the unique predicted equilibrium involves a single candidate entering at the median of the voter distribution, while the low-cost treatment has, in addition to the median-candidate equilibrium, a two-candidate equilibrium with distinct policy proposals. The only other experimental test of the citizen-candidate environment that we are aware of has been conducted by ourselves (Elbittar and Gomberg, 2009). Unfortunately, the equilibrium multiplicity turned out to be a particularly serious problem in that stdy, resulting in major coordination problems among the subjects.
    
    Our objective in this work was to design an environment, which avoids the coordination problems, while varying both cost parameters and electoral systems. In particular, in addition to the simple plurality elections, we consider the two-round runoffs. In particular, Osborne and Slivinski (1996) results imply, for certain parameter values of the model, a stronger pull for entry by politicians at the median of the voter ideal point distributions. The same pull to the center is implied by high candidate entry cost for both electoral systems. It is these implications of the model that we would like to test.
    
    Like both Cadigan (2005) and Elbittar and Gomberg (2009), we impose sincere voting, in order to concentrate on individual entry decisions by potential candidates. At the same time, we want to stay close to the large-electorate spatial model of Osborne and Slivinski (1996). To do this, while keeping the number of participants in an experimental game small, we decouple the potential candidates (whom we shall call "politicians") from the entire society of citizens. Only politicians may choose to run for office, while the set of voters (implemented in our experiments by a computer) is larger. In practice, not every voter would have name recognition and/or funding lined up to make him a viable candidate in a given election, Furthermore, only politicians are under a sufficient public scrutiny to make the assumption that their political views are known empirically plausible. In most elections, at least some of the potential "pre-candidates", though credible enough to be considered, choose not to enter the campaign. It is this entry decision that we study.
    
    Cadigan (2005) observed that the low-cost treatment results in relatively high entry by symmetric off-median subjects, compared with the high-cost treatment.  Furthermore, he notes non-negligible entry rates by candidates who, according to the model predictions, should not enter. This might seem unexpected, given that in related games of market entry, which have been extensively studied experimentally since Kahneman (1988), fast convergence to theoretically predicted entry rates of entry has been commonly observed (for a survey, see, for instance, Camerer 2003). However, Rapoport et al. (2002),in a market-entry game with asymmetric entry costs, found that subjects tend to over-enter, when the pure-strategy equilibrium implies they should be staying out, and under-enter, when the equilibrium implies they should enter. Unlike in Rappoport et al. (2002) in the citizen-candidate environment the candidate asymmetry comes not from difference in entry costs, but from a difference in their spatial location. Still, a parallel seems notable.
    
    Our own earlier obervations in Elbittar and Gomberg (2009) may be summarized as follows. Firstly, we do observe subjects reacting to treatment variable changes. In particular, both the asymmetry of politician ideal point distribution and the run-off electoral system are conducive to greater entry frequencies at the median of the voter ideal points. Secondly, we seem to confirm Cadigan's observation of comparatively high entry in situations, when equilibrium predicts no entry. In fact, entry rates remain non-negligible even from, essentially, hopeless positions. Other than that, the subjects' entry decisions seem to be reasonably close to best responding to the empirically observed entry frequencies.

The Rest of this paper is organized as follows. Section 2 presents the basic citizen-candidate model along the lines of Osborne and Slivinski (1996), section 3 describes the experimental design we emply, section 4 presents our experimental results, including the quantal response equilibrium analysis, section 5 concludes.




