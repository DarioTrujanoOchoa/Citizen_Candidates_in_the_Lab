
\section{Experimental Design and Predictions}

All experimental sessions were conducted at the Instituto Tecnologico Autonomo de Mexico (ITAM) 
in Mexico City and the subjects were undergraduates recruited in introductory economics courses. 
The experiments were computer-administered. A total of 10 experimental sessions were conducted 
and each session had between 12 and 30 participants.

In each experimental session we consecutively ran 30 elections in groups with
three candidates, with both group membership and subject ideal points
randomly changed before each election. If the total number of subjects in the
room was not divisible by 3, in each round some subjects would
be randomly selected to sit it out. Therefore, up until the last round the termination
time effectively remained random. At the beginning of each session, subjects played three practice trials.

The distribution of subjects ideal points was either constant, or varied only
once during a session, but each subject?s location was randomly chosen for each
period, which corresponded to an election. In each election subjects, having
observed their positions, had to decide whether to nominate themselves as possible
candidates. All voter decisions were taken by the computer. After each election
subjects got the feedback about the ideal points of the entrants and the
winner in their election, as well as the vote shares received by every candidate
and their own monetary payoff.

Less trials happened if participants lost all the money given or because there were a non-multiple of three; 
then, one or two participants waited until the next trial-match. 
All payments were in Mexican Pesos (MN$11 = USD$1). We started each experimental
session by allocating every agent MN140 pesos of initial capital, to
which the payments corresponding to the model parameter values were added
and subtracted. Participants were allowed to continue until 
they finished a trial with negative balance.

All six experimental treatments had three candidates with different ideal points from the interval $[0, 100]$ 
which represents the sincere voters. The six different games were built from 2 entry costs, 
2 voting rules and 3 ideal points sets. Table \ref{tab:parameters} summarizes each game parameters. 
The last three columns show the ideal points labeled \emph{Left}, \emph{Center} and \emph{Right} according 
with the relative position held by each candidate. The last four game have the same ideal position and were 
built from crossing costs $5$ and $20$ with the two voting rules. 

\begin{table}[!htbp] \centering 
	\caption{Parameters in the six treatments} 
	\label{tab:parameters} 
	\begin{tabular}{@{\extracolsep{5pt}} ccccccccc} 
		\\[-1.8ex]\hline 
		\hline \\[-1.8ex] 
		Games & VotingRule & $\alpha$ & $c$ & $b$ & $D$ & Left & Center & Right \\ 
		\hline \\[-1.8ex] 
		PLCS & Plurality Rule & $0.100$ & $5$ & $25$ & $40$ & $30$ & $50$ & $70$ \\ 
		PLCA & Plurality Rule & $0.100$ & $5$ & $25$ & $40$ & $30$ & $50$ & $80$ \\ 
		PL & Plurality Rule & $0.100$ & $5$ & $25$ & $40$ & $20$ & $30$ & $80$ \\ 
		PH & Plurality Rule & $0.100$ & $20$ & $25$ & $40$ & $20$ & $30$ & $80$ \\ 
		RL & Run-Off & $0.100$ & $5$ & $25$ & $40$ & $20$ & $30$ & $80$ \\ 
		RH & Run-Off & $0.100$ & $20$ & $25$ & $40$ & $20$ & $30$ & $80$ \\ 
		\hline \\[-1.8ex] 
	\end{tabular} 
\end{table} 


The Nash equilibria in pure strategies are stated in table \ref{tab:equilibria}. The only games with two-candidate equilibrium are \emph{PLCS} and \emph{PL}, which are those with low cost, extreme ideal points symmetric around median and voting  plurality rule. The closest equilibrium to empirical proportion of entering are marked with an asterisk. 

\begin{table}[!htbp] \centering 
	\caption{Possible equilibria with the ideal points of entering candidates}
	\label{tab:equilibria} 
	\begin{tabular}{@{\extracolsep{5pt}} ccc} 
		\\[-1.8ex]\hline 
		\hline \\[-1.8ex] 
		Games & OneCandidate & TwoCandidate \\ 
		\hline \\[-1.8ex] 
		
		PL & \textbf{30*} & $20$, $80$ \\ 
		PH & \textbf{30*} &  \\ 
		RL & \textbf{30*} &  \\ 
		RH & \textbf{30*} &  \\ 
		PLCS & $50$ & \textbf{30*, 70*} \\ 
		PLCA & \textbf{50*} &  \\ 
		\hline \\[-1.8ex] 
	\end{tabular} 
\end{table} 

